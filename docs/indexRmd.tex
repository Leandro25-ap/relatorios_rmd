% Options for packages loaded elsewhere
\PassOptionsToPackage{unicode}{hyperref}
\PassOptionsToPackage{hyphens}{url}
%
\documentclass[
]{article}
\usepackage{lmodern}
\usepackage{amssymb,amsmath}
\usepackage{ifxetex,ifluatex}
\ifnum 0\ifxetex 1\fi\ifluatex 1\fi=0 % if pdftex
  \usepackage[T1]{fontenc}
  \usepackage[utf8]{inputenc}
  \usepackage{textcomp} % provide euro and other symbols
\else % if luatex or xetex
  \usepackage{unicode-math}
  \defaultfontfeatures{Scale=MatchLowercase}
  \defaultfontfeatures[\rmfamily]{Ligatures=TeX,Scale=1}
\fi
% Use upquote if available, for straight quotes in verbatim environments
\IfFileExists{upquote.sty}{\usepackage{upquote}}{}
\IfFileExists{microtype.sty}{% use microtype if available
  \usepackage[]{microtype}
  \UseMicrotypeSet[protrusion]{basicmath} % disable protrusion for tt fonts
}{}
\makeatletter
\@ifundefined{KOMAClassName}{% if non-KOMA class
  \IfFileExists{parskip.sty}{%
    \usepackage{parskip}
  }{% else
    \setlength{\parindent}{0pt}
    \setlength{\parskip}{6pt plus 2pt minus 1pt}}
}{% if KOMA class
  \KOMAoptions{parskip=half}}
\makeatother
\usepackage{xcolor}
\IfFileExists{xurl.sty}{\usepackage{xurl}}{} % add URL line breaks if available
\IfFileExists{bookmark.sty}{\usepackage{bookmark}}{\usepackage{hyperref}}
\hypersetup{
  pdftitle={Relatorio Parcial},
  pdfauthor={Leandro Alves},
  hidelinks,
  pdfcreator={LaTeX via pandoc}}
\urlstyle{same} % disable monospaced font for URLs
\usepackage[margin=1in]{geometry}
\usepackage{color}
\usepackage{fancyvrb}
\newcommand{\VerbBar}{|}
\newcommand{\VERB}{\Verb[commandchars=\\\{\}]}
\DefineVerbatimEnvironment{Highlighting}{Verbatim}{commandchars=\\\{\}}
% Add ',fontsize=\small' for more characters per line
\usepackage{framed}
\definecolor{shadecolor}{RGB}{248,248,248}
\newenvironment{Shaded}{\begin{snugshade}}{\end{snugshade}}
\newcommand{\AlertTok}[1]{\textcolor[rgb]{0.94,0.16,0.16}{#1}}
\newcommand{\AnnotationTok}[1]{\textcolor[rgb]{0.56,0.35,0.01}{\textbf{\textit{#1}}}}
\newcommand{\AttributeTok}[1]{\textcolor[rgb]{0.77,0.63,0.00}{#1}}
\newcommand{\BaseNTok}[1]{\textcolor[rgb]{0.00,0.00,0.81}{#1}}
\newcommand{\BuiltInTok}[1]{#1}
\newcommand{\CharTok}[1]{\textcolor[rgb]{0.31,0.60,0.02}{#1}}
\newcommand{\CommentTok}[1]{\textcolor[rgb]{0.56,0.35,0.01}{\textit{#1}}}
\newcommand{\CommentVarTok}[1]{\textcolor[rgb]{0.56,0.35,0.01}{\textbf{\textit{#1}}}}
\newcommand{\ConstantTok}[1]{\textcolor[rgb]{0.00,0.00,0.00}{#1}}
\newcommand{\ControlFlowTok}[1]{\textcolor[rgb]{0.13,0.29,0.53}{\textbf{#1}}}
\newcommand{\DataTypeTok}[1]{\textcolor[rgb]{0.13,0.29,0.53}{#1}}
\newcommand{\DecValTok}[1]{\textcolor[rgb]{0.00,0.00,0.81}{#1}}
\newcommand{\DocumentationTok}[1]{\textcolor[rgb]{0.56,0.35,0.01}{\textbf{\textit{#1}}}}
\newcommand{\ErrorTok}[1]{\textcolor[rgb]{0.64,0.00,0.00}{\textbf{#1}}}
\newcommand{\ExtensionTok}[1]{#1}
\newcommand{\FloatTok}[1]{\textcolor[rgb]{0.00,0.00,0.81}{#1}}
\newcommand{\FunctionTok}[1]{\textcolor[rgb]{0.00,0.00,0.00}{#1}}
\newcommand{\ImportTok}[1]{#1}
\newcommand{\InformationTok}[1]{\textcolor[rgb]{0.56,0.35,0.01}{\textbf{\textit{#1}}}}
\newcommand{\KeywordTok}[1]{\textcolor[rgb]{0.13,0.29,0.53}{\textbf{#1}}}
\newcommand{\NormalTok}[1]{#1}
\newcommand{\OperatorTok}[1]{\textcolor[rgb]{0.81,0.36,0.00}{\textbf{#1}}}
\newcommand{\OtherTok}[1]{\textcolor[rgb]{0.56,0.35,0.01}{#1}}
\newcommand{\PreprocessorTok}[1]{\textcolor[rgb]{0.56,0.35,0.01}{\textit{#1}}}
\newcommand{\RegionMarkerTok}[1]{#1}
\newcommand{\SpecialCharTok}[1]{\textcolor[rgb]{0.00,0.00,0.00}{#1}}
\newcommand{\SpecialStringTok}[1]{\textcolor[rgb]{0.31,0.60,0.02}{#1}}
\newcommand{\StringTok}[1]{\textcolor[rgb]{0.31,0.60,0.02}{#1}}
\newcommand{\VariableTok}[1]{\textcolor[rgb]{0.00,0.00,0.00}{#1}}
\newcommand{\VerbatimStringTok}[1]{\textcolor[rgb]{0.31,0.60,0.02}{#1}}
\newcommand{\WarningTok}[1]{\textcolor[rgb]{0.56,0.35,0.01}{\textbf{\textit{#1}}}}
\usepackage{graphicx,grffile}
\makeatletter
\def\maxwidth{\ifdim\Gin@nat@width>\linewidth\linewidth\else\Gin@nat@width\fi}
\def\maxheight{\ifdim\Gin@nat@height>\textheight\textheight\else\Gin@nat@height\fi}
\makeatother
% Scale images if necessary, so that they will not overflow the page
% margins by default, and it is still possible to overwrite the defaults
% using explicit options in \includegraphics[width, height, ...]{}
\setkeys{Gin}{width=\maxwidth,height=\maxheight,keepaspectratio}
% Set default figure placement to htbp
\makeatletter
\def\fps@figure{htbp}
\makeatother
\setlength{\emergencystretch}{3em} % prevent overfull lines
\providecommand{\tightlist}{%
  \setlength{\itemsep}{0pt}\setlength{\parskip}{0pt}}
\setcounter{secnumdepth}{-\maxdimen} % remove section numbering

\title{Relatorio Parcial}
\author{Leandro Alves}
\date{28/04/2021}

\begin{document}
\maketitle

\hypertarget{titulo-do-capuxedtulo}{%
\section{Titulo do Capítulo}\label{titulo-do-capuxedtulo}}

\hypertarget{subtuxedtulo-01}{%
\subsection{Subtítulo 01}\label{subtuxedtulo-01}}

\hypertarget{sub-subtuxedtulo}{%
\subsubsection{Sub subtítulo}\label{sub-subtuxedtulo}}

\hypertarget{subsubsubtuxedtulos}{%
\paragraph{Subsubsubtítulos}\label{subsubsubtuxedtulos}}

Objetivo do trabalho foi avaliar a emissão de \(CO_{2}\) e vamos estudar
o \(Mg^{2+}\)

Atalho para gera o HTML \texttt{CONTROL+SHIFT+K}

\texttt{\{r,out.width\ =\ "100\%",fig.cap="titulo",fig.align\ =\ \textquotesingle{}center\textquotesingle{}\}\ knitr::include\_graphics("img/desvio.png")}

\hypertarget{ediuxe7uxe3o-de-fuxf3mulas}{%
\subsection{Edição de Fómulas}\label{ediuxe7uxe3o-de-fuxf3mulas}}

\[
\mu \rightarrow \text{Média Amostral: } \bar{x} = \frac{\sum \limits_{i=1}^n x_i}{n} \\
\sigma^2 \rightarrow \text{Variância Amostral: } s^2 = \frac{\sum_{i=1}^n (x_i - \bar{x} )^2}{n-1}
\]

Outra fórmula para variância.

\[
s^2 = \frac{\sum \limits_{i=1}^n x_i - \frac{\left(\sum \limits_{i=1}^n x_i \right)^2}{n} }{n-1}
\]

Raiz quadrada

\(\sqrt{\pi*2-cos45"}\)

\hypertarget{introduuxe7uxe3o-do-projeto}{%
\section{Introdução do Projeto}\label{introduuxe7uxe3o-do-projeto}}

O Brasil é o maior produtor de \emph{cana-de-açúcar} do mundo com mais
de \textbf{9 milhões} de hectares em áreas plantadas. O estado de São
Paulo é o maior produtor nacional de açúcar.

\texttt{\{r,out.width\ =\ "100\%",fig.cap="titulo",fig.align\ =\ \textquotesingle{}center\textquotesingle{}\}\ knitr::include\_graphics("https://media.giphy.com/media/ule4vhcY1xEKQ/giphy.gif")}

FONTE: \href{https:://giphy.com}{GIPHY}

\textbf{FONTE}: \href{https://www.pexels.com/}{PEXELS}

Vamos plotar a função \(y = x^2- 5x + 6\)

para colocar o chuck , selecionar \texttt{CONTROL+ALT+L}

\begin{Shaded}
\begin{Highlighting}[]
\NormalTok{x <-}\StringTok{ }\DecValTok{-10}\OperatorTok{:}\DecValTok{10}
\NormalTok{y <-}\StringTok{ }\NormalTok{x}\OperatorTok{*}\NormalTok{x }\DecValTok{-5}\OperatorTok{*}\NormalTok{x }\OperatorTok{+}\DecValTok{6}
\KeywordTok{plot}\NormalTok{(x,y,}\DataTypeTok{pch=}\DecValTok{21}\NormalTok{,}\DataTypeTok{col=}\StringTok{"black"}\NormalTok{,}\DataTypeTok{bg=}\StringTok{"gray"}\NormalTok{,}\DataTypeTok{cex=}\DecValTok{2}\NormalTok{,}\DataTypeTok{las=}\DecValTok{1}\NormalTok{)}
\end{Highlighting}
\end{Shaded}

\includegraphics{indexRmd_files/figure-latex/unnamed-chunk-1-1.pdf}

\end{document}
